\documentclass{article}
\usepackage{geometry}[margin = 1in]
\usepackage{amsmath, amssymb, mathtools}

\begin{document}
\title{Integral as a Linear Transformation}
\author{myutin}
\date{\today}
\maketitle

Suppose we have an instantaneous motion vector $ (x _0, v _0, a _0) ^T $ at time $ 0 $, with constant acceleration. Then, at time $ t $, the vector is given as:
\begin{align*}
	a _t &= a _0 ,\\
	v _t &= v _0 + t a _0 ,\\
	x _t &= x _0 + t v _0 + \tfrac{1}{2} t ^2 a _0
.\end{align*}
In other words, we have:
\[
	\begin{bmatrix}
		x _t \\ v _t \\ a _t
	\end{bmatrix}
	=
	\begin{bmatrix}
		1 & t & \tfrac{1}{2} t ^2 \\
		0 & 1 & t \\
		0 & 0 & 1
	\end{bmatrix}
	\begin{bmatrix}
		x _0 \\ v _0 \\ a _0
	\end{bmatrix}
\]
Therefore, if we start with a multivariate normal distribution:
\[ (x _0, v _0, a _0) ^T \sim {\cal N}(\mu _0, \Sigma _0) \]
Then at time $ t $, the distribution is given by $ {\cal N}({\cal J} _t \mu _0, {\cal J} _t \Sigma _0 {\cal J} _t) $
\end{document}